\chapter{Methodology}
\label{cp:Methodology}

The proposed framework uses a bottom-up pipeline to 
gradually infer a high-level representation of the scratchs and stains from low-level features in an back cover of mobile phone image. Firstly by mean of wavelet transform, the surface texture properties such as scratch and stain are decomposed into so-called wavelet characteristics. Then multivariate statistical approach, i.e. Hotelling $T^{2}$ control chart is utilized to monitor the mean vector of a multivariate process, which can be used to judge the existence of scratch defects in the sample image. 

\section{Wavelet transform decomposition}




Here is one example demonstrated by Fig.~\ref{fig:cnn} that adapted from~\cite{hijazi2015using}. 

\begin{figure}[h]
\centering
\includegraphics[width=1\textwidth]{images/cnn.pdf}
\caption[Structure of CNNs]{A general structure of CNNs, which is frequently used for image recognition. There are several layers such as input layer, convolutional layer, non-linear layer, pooling layer, and fully-connected layer. The pink squares indicate
the convolutional kernels. Figure is adapted from~\cite{hijazi2015using}.}
\label{fig:cnn}
\end{figure}



